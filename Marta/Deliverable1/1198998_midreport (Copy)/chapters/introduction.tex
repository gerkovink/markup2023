In many contexts we come across data that take the form of rankings. This type of data is present, for example, in researches concerning the preferences of individuals for different candidates in elections ( \cite{stern1993probability} \cite{koop1994rank} or for different consumer goods (\cite{beggs1981assessing}). In these cases, the aim is usually to investigate individual's choice decisions  (\cite{yu2019analysis}).  Another area where ranking data are common is in sport  and game analytics, where the goal is rather to use game results to estimate the ability of competitors. This is done for example in \cite{van2023bayesian} with Formula One race results, in \cite{ali1998probability} with horses-races or in \cite{graves2003hierarchical} with NASCAR competitions. \\
\\
If ranking data want to be analysed, models tailored for this typology of data are needed. There are several known parametric models in the literature. One of the most popular is the one proposed by \cite{luce1959possible} and \cite{plackett1975analysis}. Rank- Ordered Logit models (ROL) (\cite{hausman1987specifying}) extended it by incorporating covariates. \\
\\
Although the validity of the ROL models has been widely acknowledged, their execution can be complex in the practice for two reasons. The first one concerns the estimation, which can become computationally burdensome when the number of competitors to be ranked is large: \cite{alvo2014statistical}  report fifteen competitors as the threshold in this sense. Secondly, the interpretation of parameters is not straightforward, as it requires multiple steps to have them related back to the outcome (refer to  Algorithm \ref{alg:sim} to understand how the rank of a competitor can be recovered from his ability parameter).\\
\\
These challenges lead researchers to use alternative models for rating competitors. For example, a linear regression was used in \cite{eichenberger2009best}, a multilevel approach in \cite{bell2016formula} and a Beta regression in \cite{stern2008ranking} \\
The approach that will be investigated in this paper is based on the transformation of the rank order of competitors given in each game into a proportion of competitors beaten for each player. This outcome can then be modeled using a Beta regression with dummies variables representing competitors as predictors. The associated regression coefficients can then be used to provide the desired rating of competitors. The downside of this approach is that it makes the incorrect assumption that the achieved ranking is (conditionally) independent of the achieved rank of the other competitors in the game. For example, if an averaged-skilled runner enters a local competition, he may ends up in the best positions and achieve a high proportion of competitors beaten. Nonetheless, if the same athlete enters a national level race where only the best contenders compete, his performance will be worse and so its proportion of competitors beaten.\\
\\
In summary, the Rank-Ordered Logit model and  Beta regression both have advantages and disadvantages in their use. Which one is preferable and when is currently unknown. In this project we will try to evaluate this and therefore answer the following research question: under which conditions (if any) is a proportion model a reasonable approximation for a complex rank-ordered model? \\
\\
This report is structured as follows: first, in Chapter 2  additional background about models for ranking data is provided and Plackett-Luce model is introduced and explained. In the same chapter we also discuss the use of alternative  models, with a particular focus on Beta regression. In Chapter 3, the simulation study  used to answer the research question is delineated, alongside with the  methodology used to compare the two models. In Chapter 4, we present simulations results and insights while in Chapter 5 we apply both models to a real-world example for illustration purposes. We end the paper with conclusions and suggestions for researchers who want to perform inference on ranking data.
